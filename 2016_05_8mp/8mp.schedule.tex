\section{Schedule}
\label{sec:schedule}
\begin{itemize}[leftmargin=*]
\item Year 1: 
\begin{itemize}[leftmargin=*]
\item Months 1-4: Get data from Orbis. Set up server.
\item Months 4-7: Analyze data quality and bias.
\item Months 8-12: (i) Study the effects of data quality (ii) Set up a comprehensive database at the city level (iii) (side project) Analyze inequality in last names in the USA. (iv) (collaboration) Examine the role of offshore centers in economy. 
At the end of the 12th month I expect to have developed a paper on how to assess data quality and its effect, as well as a paper describing inequality among directorates in the USA.
\end{itemize}
\item Year 2:  
During the second year, we will develop the project on the effect of interlocks on structural transformation. 
Based on the database on cities and the database on companies, 
we can start investigating the relative effect of interlocks, city characteristics and institutions on structural transformation.
The steps required for this project will be: (i) Create the product-service space and analyze its relationship with structural transformation (ii) Model the relationship between interlocks and product-service space and what other factors affect it (causal inference) (iii) Analyze if interlocks affect structural transformation by fostering innovation, or collaboration, and how does it depend on city characteristics and institutions.

Moreover, 
we will develop a paper showing how understanding the bipartite nature of the system (people in companies) can help reconciliate contradictory findings in corporate governance.
\medskip
\item Year 3 and 4:
The schedule for year 3 and 4 will depend on the findings during the previous years. 
If the results from year 2 are interesting,
we can investigate the origins of interlocks.
For example, previous research has investigated the factors that increase the probability of interlock either at the company level or at the director level.
We can model the process at the bipartite level in order to determine which factors \textit{actually} have an effect.
Otherwise, we can transition to other projects in the area of corporate governance,
for example understanding the effect of ownership, shared directors and variations accross economic activities and region in the projects of section~\ref{sec:other}.
\end{itemize}
