\section{Schedule}
\label{sec:schedule}
\begin{itemize}[leftmargin=*]
\item Year 1: 
\begin{itemize}[leftmargin=*]
\item Months 1-4: Get data from Orbis. Set up server.
\item Months 4-7: Analyze data quality and bias.
\item Months 8-12: (i) Study the effects of data quality. (ii) Set up a comprehensive database at the city level (iii) (side project) Analyze inequality in last names in the USA. (iv) (collaboration Examine the role of offshore centers in economy. 
At the end of the 12th months I expect to have developped a paper on how to assess data quality and its effect, as well as a paper describing inequality among directorates in the USA.
\end{itemize}
\item Year 2:  
During the second year, we will develop the project on the effect of interlocks in structural transformation. 
Based on the database on cities and the database on companies, 
we can start investigating the relative effect of interlocks, city characteristics and institutions in structural transfromation.
The steps required for this project will be:
\begin{itemize}[leftmargin=*]
\item Create the product-service space and analyze its relationship with structural transformation.
\item Model the relationship between interlocks and product-service space and what other factos affect it (causal inference).
\item Analyze if interlocks affect structural transformation by fostering innovation, or collaboration, and how does it depend on city characteristics and institutions.
\end{itemize}
Moreover, 
we will develop a paper showing how understanding the bipartite nature of the system (people in companies) can reconciliate empirical findings in corporate governance.
\item Year 3 and 4:
The schedule for year 3 and 4 depends on the findings on previous years. 
If the results from year 2 are interesting,
we can investigate the origins of interlocks.
For example, previous research have focused on the factors either at the company level or at the director level that increase the probability that an interlock.
However, modeling the process at the bipartite level can bring new insights.

Otherwise, we can transition to other projects in the area of corporate governance,
for example understanding the effect of ownership, shared directors and sector and space variation in the projects of section~\ref{sec:other}.
\end{itemize}


