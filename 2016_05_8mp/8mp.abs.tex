\section{Abstract}
\label{sec:description}
The study of corporate networks dates back to the beginning of the 20$th$ century~\citep{jeidels1905,Lenin1917},
when Jeidels and Lenin noticed an increase in the relationships between banks and industry. 
These relationships were created through interlocking directorates -- the appointment of directors with positions in both industries.
Nowadays, 25\% of the top one million companies worldwide are connected by interlocks (Fig. \ref{fig:fig1}).
However, not much is know about the consequences of shared directors.
In the past decades, researchers have investigated the effect of interlocks in firm performance, innovation, acquisitions, mergers, capital growth, firm reputation, and adoption of structures and strategies (for a review see~\cite{Mizruchi1996}).
In spite of the extensive literature,  
only the spread of structures and strategies has been consistently associated to interlocking directorates\citep{Haunschild1993,Davis1997,Davis1991,Rao1999},
while the other consequences of interlocks are still controversial~\cite{Mizruchi1996}.
These contradictory results can be attributed to biases in the methods and data used in the studies,
namely a few dozen companies in a particular sector and country.
Using data from the Orbis database, 
comprising 200 million companies and 100 million directors, 
we develop a theoretical and methodological framework to answer the question:
``Do (and if so, how) interlocks facilitate structural transformation?'',
where structural transformation is the reallocation of economic activity across the broad sectors agriculture, manufacturing and services.

Firstly, we will create a network of related economic activities (sectors),
where two economic activities are closer in the network if companies from both sectors are often co-located in the same city.
Similarly to previous research at the country level~\cite{hidalgo2007, hausmann2011, Hausmann2006,hidalgo2009}, 
we expect that structural transformation occurs through the development of new sectors that are close in the network to the existing sectors.
Moreover, we anticipate that this diffusive process will be a predictor of economic growth~\cite{hidalgo2009}.
Secondly, we will quantify to what extent the presence of interlocks facilitates this diffusive process,
as well as the relative weight of institutions and city characteristics..
Finally, we will test if interlocks facilitate diffusion trough an increase in collaboration and innovation between companies.


This research proposal is organized as follows. 
In section~\ref{sec:question}, we focus on the research questions that we will study,
and explain the propositions, concepts and hypotheses of the project.
In section~\ref{sec:complexity}, we give an overview of complexity methods that have been applied to social sciences.
Finally, we describe the data in section~\ref{sec:data}.



