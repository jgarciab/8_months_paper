\section{Brief description of the project}
\label{sec:description}
The study of corporate networks dates back to the beginning of the 20$th$ century~\citep{Lenin1999},
when researchers noticed increased relationships between banks and industry.
These relationships were created through the appointment of directors with positions in both industries.
Nowadays, 25\% of the companies worldwide are connected by shared board directors (interlocks).
Several explanations (micro-motives) have been pointed out for the creation of such links. 
The micro-motives include facilitating collusion (cooperate to limit competition), 
cooptation (absorption of potentially disruptive elements), 
creating legitimacy by hiring respectable directors, 
career advancement and social cohesion.


The presence of interlocks have been correlated with macro-outcomes, 
finding contradictory results with profits, and with the effect in merges and takeover~\citep{Mizruchi1996}.
Only the spread of corporate strategies has been consistently found in the literature~\citep{Mizruchi1996}. 
The ambiguous results can be attributed to the biased data studied in the past,
namely a few dozen companies in a particular sector.
Using data from the Orbis database, 
comprising 200 million companies and 100 million directors, 
we will analyze the effects.

Moreover, the fine allows us to study the characteristics of the network without imposing a national character. 
In this way blablbah (d. In some regions the business communities are organized along
national borders, whereas in other areas the locus of organization is at the
city level or international level.) 


This research proposal is organized as follows. 
Firstly we explain the characteristics of the data.
Secondly we give an overview of complexity methods that can and have been applied to social sciences.
Thirdly, we provide an overview of the problems (research questions) within the project that can be tackled using complex systems methods.
Finally, we focus on the propositions, concepts and hypotheses of the project.