\documentclass[pre,twocolumn,twoside,byrevtex,superscriptaddress,floatfix]{revtex4-1}

\usepackage{currfile}
\lefthyphenmin=3
\righthyphenmin=2

\usepackage{graphicx,epsfig,verbatim,enumerate}
\usepackage{amssymb,amsmath}
\usepackage{ifthen}

\usepackage{longtable}

\usepackage{mathtools}

\newboolean{twocolswitch}

\newcommand{\avg}[1]{\left\langle#1\right\rangle}
\newcommand{\tavg}[1]{\langle#1\rangle}

\newcommand{\LavgHa}[2]{\tavg{L_{#1,#2}}}
\newcommand{\LavgHaN}[2]{\tavg{L_{#1,#2}}}
\newcommand{\Lavg}{\tavg{L}}

\newcommand{\sindex}[1]{}
\newcommand{\nindex}[1]{}

\newcommand{\etal}{\textit{et al.}}
\newcommand{\www}[1]{\url{#1}}
\newcommand{\req}[1]{(\ref{#1})}
\newcommand{\Req}[1]{Eq.~(\ref{#1})}

% Lettrines
\usepackage{lettrine}

\newcommand{\yes}{}
\newcommand{\no}{}
\newcommand{\tbf}{\textbf}
\newcommand{\tit}{\textit}

\newcommand{\dee}[1]{\mbox{d}#1}
\newcommand{\pdiff}[2]{\frac{\partial #1}{\partial #2}}
\newcommand{\pdiffsq}[2]{\frac{\partial^2 #1}{{\partial #2}^2}}
\newcommand{\diff}[2]{\frac{{\rm d}#1}{{\rm d}#2}}
\newcommand{\diffsq}[2]{\frac{{\rm d}^{2}#1}{{\rm d} {#2}^2}}
\newcommand{\tdiff}[2]{\mbox{d} #1/\mbox{d} #2}
\newcommand{\tdiffsq}[2]{\mbox{d}^{2} #1/\mbox{d} {#2}^2}
\newcommand{\tpdiff}[2]{\partial #1/\partial #2}
\newcommand{\tpdiffsq}[2]{\partial^2 #1/\partial {#2}^2}
\newcommand{\postdee}[1]{\,\mbox{d}#1}

\usepackage{color}
\newcommand{\todo}[1]{\noindent\textcolor{blue}{{$\Box$ #1}}}

\newcommand{\Prob}[1]{{\rm Pr}\left(#1\right)}

% \newcommand{\kstar}{d^{\ast}}
% \newcommand{\kstari}{k_i^\ast}
\newcommand{\kstar}{d^\ast}
\newcommand{\kstari}{d_i^\ast}

\newcommand{\dstar}{d^\ast}
\newcommand{\dstari}{d_i^\ast}

\newcommand{\phifix}{{\phi^{\ast}}}
\newcommand{\phifixc}{{\phi_c^{\ast}}}
\newcommand{\phifixb}{{\phi_b^{\ast}}}

\newcommand{\phiiactive}{\phi_{i,t}}

\newcommand{\phiiup}{{\phi_{i,\rm on}}}
\newcommand{\phiidown}{{\phi_{i,\rm off}}}

\newcommand{\phiup}{{\phi_{\rm on}}}
\newcommand{\phidown}{{\phi_{\rm off}}}

\newcommand{\Gfun}{G}

\newcommand{\dstardist}{g}
\newcommand{\dosedist}{f}
\newcommand{\dosedistk}{f^{k\star}}

% chaotic contagion
\newcommand{\edgeinfprob}{\rho}
\newcommand{\nodeinfprob}{\phi}

\newcommand{\avgdegree}{k_{\rm avg}}

\newcommand{\stateA}{S_{0}}
\newcommand{\stateB}{S_{1}}

\newcommand{\lcce}{H}

\newcommand{\effectivecount}{f_{q,\textnormal{exp}}}
\newcommand{\effectivecountq}[1]{f_{#1,\textnormal{exp}}}

\newcommand{\simonrho}{\rho}
\newcommand{\zipfexponent}{\alpha}

\newcommand{\partitionprob}{q}
\newcommand{\partitiontype}[1]{$\partitionprob$$=$$#1$}

\newcommand{\onehalf}{\frac{1}{2}}
\newcommand{\onequarter}{\frac{1}{4}}

\newcommand{\clgamma}{\hat{\zipfexponent}}
\newcommand{\xmin}{r_{\textnormal{max}}}
\newcommand{\kstat}{D}
\newcommand{\clp}{\mbox{$p$-value}}
\newcommand{\sigamma}{1-\simonrho}

\setboolean{twocolswitch}{true}

\begin{document}

\title{\protect
Quantitative patterns in drone wars
}

\author{
\firstname{Javier}
\surname{Garcia-Bernardo}
}
\email{javier.garcia-bernado@uvm.edu}

\affiliation{Department of Computer Science,
  The University of Vermont,
  Burlington, VT 05401.}

\author{
\firstname{Peter Sheridan}
\surname{Dodds}
}
\email{peter.dodds@uvm.edu}

\affiliation{Department of Mathematics \& Statistics,
  Vermont Complex Systems Center,
  Computational Story Lab,
  \& the Vermont Advanced Computing Core,
  The University of Vermont,
  Burlington, VT 05401.}

\author{
\firstname{Neil F.}
\surname{Johnson}
}
\email{njohnson@physics.miami.edu}

\affiliation{Department of Physics,
University of Miami
Coral Gables, FL 33124.
}

\date{\today}

\begin{abstract}
  \protect
  Attacks by drones (i.e., unmanned combat air vehicles) continue to
generate heated political and ethical debates. Here we examine instead
the quantitative nature of drone attacks, focusing on how their
intensity and frequency compares to other forms of human conflict.
In contrast to the power-law distribution found recently for insurgent
and terrorist attacks, we find that the severity of attacks is well
described by a lognormal distribution, indicating that the dynamics
underlying drone attacks differ from these other forms of human
conflict.
Likewise, the pattern in the timing of attacks is consistent with one
side having almost complete control.
We show that these novel features can be reproduced and understood
using a generative mathematical model in which resource allocation to
the dominant side is regulated through a feedback loop.

 
\end{abstract}

\pacs{89.65.-s,89.75.Da,89.75.Fb,89.75.-k}

\maketitle

\section*{Introduction}
Dating back to physicist L.~F.~Richardson's pioneering
work nearly 100 years ago~\cite{Richardson194}, the quantitative
analysis of human conflict has attracted research interest from across
the social, biological, economic, mathematical and physical
sciences~\cite{Morgenstern2013,Cederman2003,Clauset2009,Clauset,Bohorquez2009a,Dedeo2011}.
As in other human activities~\cite{Barabasi2005,Gabaix2003}, power
laws have been identified in the severity distribution of individual
attacks in insurgencies and
terrorism~\cite{Johnson2013b,Clauset2009,Clauset,Bohorquez2009a}, and
in the temporal trend in
events~\cite{Johnson2013b,Johnson2011a,Clauset}.  These studies found
that across a wide range of insurgent wars in which a relatively small
opponent such as an insurgency (Red Queen~\cite{Johnson2011a}) fights
a larger one such as a state (Blue King~\cite{Johnson2011a}), the
probability distribution for the severity $s$---the number of fatalities---of an event (i.e., clash
or attack) is given by $P(s) \propto s^{-\alpha}$ where $\alpha \sim
2.5$, while the trend in the timing of attacks is given by $\tau_{n}
= \tau_1 n^{-b}$\textcolor{red}{,} where $\tau_n$ is the time interval
between events $n$ and $n+1$, $n=1,2,\dots$ and $b$ is the escalation
parameter. When $b=0$, the Blue King and Red Queen are evenly matched,
with both effectively running on the same spot---hence the terminology
surrounding the Red Queen~\cite{Johnson2011a}. When $b\neq 0$, there
is an escalation in the frequency of attacks which can be interpreted
as a relative advantage between the Blue King and the Red
Queen~\cite{Johnson2011a}. The power-law finding for the distribution
of event severities is consistent with the Red Queen (i.e., insurgent
force) evolving dynamically as a self-organizing system composed of
cells (i.e., clusters) that sporadically either fragment under the
pressure of the Blue King (e.g., state) or coalesce to create larger
cells, and taking the severity of attacks as proportional to the sizes
of the resulting cells~\cite{Bohorquez2009a}.

This paper examines, for the first time, event patterns in the new
form of human conflict offered by unmanned combat air vehicles
(drones)~\cite{wars}. We focus on Pakistan and Yemen because of their
association with drone strike campaigns, using data from the New
America Foundation (NATSEC) and the South Asia Terrorism Portal (SATP)
databases. The situation of drone wars differs from the typical
situation for insurgencies and terrorism in that the attacks are now
carried out by the Blue King on the Red Queen. Moreover, the
sophistication of the action-at-a-distance technology means that any
delay in the Blue King's next attack is likely to have come from a
constraint within Blue itself (e.g., political opposition) as opposed
to any direct counter-adaptation by the Red Queen. Indeed, we find
that the drone attacks do not follow the universal patterns in the
severity and timing for insurgencies and terrorism. Instead our
analysis indicates a new regime of human conflict in which the Blue
King has almost complete control over the conflict. We develop a
generative model in which the severity and timing of attacks are
determined solely by the resources of the Blue King, but are regulated
by a positive feedback loop due to the Blue King's internal
sociopolitical and economic constraints. We show that this simple
model reproduces the main features of the original data and hence
yields novel insight into the nature of drone warfare.

\begin{figure*}[tp!]
  \centering	
  \includegraphics[width=\textwidth]{drawing-2.pdf}  
  \caption{
    \textbf{The severity of drone attacks approximately follows a lognormal
      distribution.}
    Complementary Cumulative Distribution Function
    (CCDF) of the severity of attacks (blue dots and solid line) and
    best fits (dashed line) to power-law (A, C) and lognormal (B,
    D) distributions for drone attacks in Pakistan (A, B) and Yemen
    (C, D). The optimal parameters for each distribution are (A)
    $\alpha = 4.82$, (B) $\mu = 1.60$, $\sigma = 0.64$, (C)
    $\alpha = 2.21$ and (D) $\mu = 1.65$, $\sigma = 0.77$.
    Bottom panels (E, F) show the severity of attack (vertical lines,
    left axis) and escalation parameter $b$ (right axis) for a
    moving window of 50 attacks in Pakistan (E) and Yemen (F).
  }
  \label{fig:drones.data}
\end{figure*}

\begin{figure*}[tp!]
  \centering
  \includegraphics[width=\textwidth]{drawing-1.pdf}
  \caption{
    \textbf{A simple model of drone wars.}
    (A) The Blue King's
    resources (funding, units, experience, etc.) influence the
    frequency and the magnitude of attacks. Resources are
    degraded continuously, with some being invested to create a
    positive feedback loop. The civilian population and other
    variables, summarized as \textit{I}, influence the strength
    of this positive feedback. The dashed line indicates that
    the Red Queen (i.e., target group) is only able to affect the
    Blue King's attacks indirectly. The amount of resources
    available corresponds to \textit{A}, the advantage of the
    Blue side.
    (B) Our simple model has one unstable fixed point at
    \textit{A} = 0, given that $\alpha \cdot I > \lambda$. (C) A
    `resistance' is added to the model due to internal Blue King
    constraints. The model now has a new fixed point at $A =
    \frac{\alpha I - \lambda}{\beta}$. (D-F) Our model is able
    to replicate the observed findings. Complementary Cumulative
    Distribution Function (CCDF) for event severity (blue dots
    and solid line) and the best fits (dashed line) to power-law
    (D, $\alpha = 2.7519$) and lognormal (E, $\mu=1.7$,
    $\sigma=0.83$) distributions for the model. (F) Severity of
    attack (vertical lines, left axis) and escalation parameter
    $b$ for a moving window of 50 attacks (right axis) for the
    model. Note the resemblance to Fig.~\ref{fig:drones.data}.
  }
  \label{fig:drones.model}
\end{figure*}

\section*{Results and Discussion}
Figs.~\ref{fig:drones.data}A--D show the complementary cumulative distribution function
(CCDF) of the severity and frequency of drone attacks using the NATSEC
database. We fit power-law and lognormal 
distributions (dashed green and solid black lines respectively; see methods)
for attacks in Pakistan (Figs.~\ref{fig:drones.data}A--B) and 
Yemen (Figs.~\ref{fig:drones.data}C--D).
We find
that the severity of the strikes is approximately described by lognormal
distributions,
in contrast to that expected for attacks carried out by terrorist and
insurgent groups which follow a power-law distribution reflecting
their underlying cluster sizes~\cite{Johnson}.
Our observation is most well supported for Pakistan,
and tentative for Yemen for which we have far less data,
and where the best fit lognormal 
is less sympathetic with larger events.
This finding of lognormality is
consistent with the notion that a drone has a specific design which
tends to pre-determine the order of magnitude of its range of
destruction and likely severity, in sharp contrast to attacks by a
terrorist or insurgent cluster where the severity of attack can vary
greatly according to the size of the cluster carrying out that
attack. For drone attacks, a lognormal distribution will then arise
naturally from a sum of normal distributions with variable mean and
standard deviation, which in turn reflects variations away from the
pre-determined impact due to differences in the terrain and the number
of people in the target group.

We now turn to
the timing of attacks in order to gain insight into the temporal
dynamics of the Blue King-versus-Red Queen activity. Following
previous work~\cite{Johnson2013b}, we plot the time interval between
consecutive attacks $\tau_n$ as a function of the cardinal number of
the attack $n=1,2,3,\dots$. The escalation parameter $b$ is the
exponent of the power-law fit $\tau_n = \tau_1 n^{-b}$, which will be
the slope of the best-fit line on a double logarithmic plot. In the
organizational development literature in which progressive events are
related to production, this is referred to as a development curve
while in the psychology literature, where subsequent events correspond
to completing a certain task, it is referred to as a learning 
curve~\cite{Johnson2011a}---in this sense, the `production' or `completion'
of drone attacks has a natural connection to human activity in these
wider fields.

For both Pakistan and Yemen, we find that the
parameter $b$ fails to stabilize around 0 
(Figs.~\ref{fig:drones.data}E--F).
Instead, the
drone attacks exhibit an initial large escalation (i.e., large positive
$b$) which then transitions to a de-escalation (i.e., large negative
$b$). Given the difficulty for the Red Queen to thwart drone attacks
directly, Figs.~\ref{fig:drones.data}E--F suggest that one side (Blue King) effectively
holds complete control for an extended period of time, and that some
internal constraints then arise within the Blue King entity that
eventually de-escalate the drone attacks.

Fig.~\ref{fig:drones.model}A shows our
simple model for drone attacks. In our model, the Blue King possesses
certain resources, for example experience, units, and funding. These
resources degrade over time if no investment is made in the Blue
King's activity, i.e., if the government does not invest in its own
drone development or information research. We assume that if the
investment (i.e., funding, time, etc.) is higher than the degradation
rate $\lambda$, then the available resources increase due to a
positive feedback loop (Fig.~\ref{fig:drones.model}B). This feedback loop has been proposed
in models of conventional terrorism~\cite{Clauset2012}, and can be
affected by external agents, for example public opinion or by the
perceived number of Red Queen cells remaining that need targeting. The
resources available at any particular time will affect the timing and
severity of the next drone attack. For simplicity we take the
frequency of attacks as directly proportional to the resource level,
while the severity of the attack depends on the available resources
and the characteristic of the terrorist cells (see methods). We add
`resistance' to the model to mimic constraints within the Blue King,
for example due to political opposition, and hence obtain the logistic
equation which has a new stable equilibrium at a non-zero value, as
shown in Fig.~\ref{fig:drones.model}C. 

This simple model is able to replicate the drone strike data
(Figs.~\ref{fig:drones.model}D--F). The escalation parameter $b$ is proportional to
$(\alpha\cdot I - {\lambda})$, which means that the Blue King has a
positive advantage if the investment term is larger than the
degradation term (Fig.~\ref{fig:drones.model}C). A constant $b>0$, corresponding to the
frequency of attacks increasing continuously, is achieved if the
resources increase exponentially---hence this is only sustainable for
short periods of time. A constant $b<0$, corresponding to the
frequency of attacks decreasing continuously, is achieved if the
resources decrease exponentially, i.e., when there is little or no
investment. Assuming each drone acts individually and the attack
severity varies slowly with the available resources, which is
consistent with increases in precision requiring significant amounts
of development effort (see methods), we are able to recreate the
lognormal distribution for the severity of attacks.

In summary, our analysis reveals new patterns in the severity and
timing of attacks in drone wars, which themselves represent a new form
of action-at-a-distance human conflict. We have purposely stepped
aside from issues of ethics or technology, choosing instead to focus
on the event data since they represent a quantitative measure of drone
war impact. We have shown that a simple model, in which the production of
drones evolves from a shared pool of resources controlled by a
feedback loop, is able to recreate the original data and therefore
explain the overall dynamics of the Blue King's drone campaign.
Going forward, our model could be also used to 
explore how wars might 
unfold when drones are used 
by two or more sides in conflicts.

\section*{Methods}
We obtained all data from the NATSEC database:
\url{http://natsec.newamerica.net/} and crosschecked with the SATP
database:
\url{http://www.satp.org/satporgtp/countries/pakistan/}.

We obtained the best fit to power-law distributions following Clauset et
al.~\cite{Clauset2009}. We fitted lognormal distributions using the
maximum likelihood estimators. For the learning rate analysis, $\tau_n
= \tau_1 n^{-b}$, we plotted the number of attack vs. the time between
attacks on a log-log scale. We accepted every value of $b$ that
allowed for a correlation greater than 20\%, which allows us to
measure fast transitions.

We simulated 100 attacks with our
model. The initial advantage was set to $10^{-3}$. The time to the
next attack is equal to $A^{-1}$. At every step (attack) the advantage
of the Blue King changed by the factor 
$\alpha\cdot I - {\lambda}$.
For the first 50 attacks 
$\lambda = \alpha\cdot I = 1.075$; 
for the last 50 attacks 
$\lambda = 1.075$ 
and 
$\alpha\cdot I = 0$.
We model the attacks as individual drones strikes; each individual
attack was drawn from a zero-truncated normal distribution, whose mean
and standard deviation were set to be equal to the root square of the
largest known Red group. The area of a self-avoiding random walk in
two dimensions increases as $A \propto n^{1/2}$, where n is the number
of individuals in the plane~\cite{Havlin1982}; the magnitude of the
attacks is fixed, hence the number of casualties scales with
$n^{1/2}$. The number of known Red groups was set to $(100+10^4\cdot
A) = [110,500]$ and they were assumed to be power-law distributed with
$\alpha =2.5$~\cite{Bohorquez2009a,Richardson194}.  

\acknowledgments
We are grateful for funding from the Vermont Complex
Systems Center.
PSD was supported by NSF CAREER Grant No. 0846668.

%merlin.mbs apsrev4-1.bst 2010-07-25 4.21a (PWD, AO, DPC) hacked
%Control: key (0)
%Control: author (8) initials jnrlst
%Control: editor formatted (1) identically to author
%Control: production of article title (-1) disabled
%Control: page (0) single
%Control: year (1) truncated
%Control: production of eprint (0) enabled
\begin{thebibliography}{15}%
\makeatletter
\providecommand \@ifxundefined [1]{%
 \@ifx{#1\undefined}
}%
\providecommand \@ifnum [1]{%
 \ifnum #1\expandafter \@firstoftwo
 \else \expandafter \@secondoftwo
 \fi
}%
\providecommand \@ifx [1]{%
 \ifx #1\expandafter \@firstoftwo
 \else \expandafter \@secondoftwo
 \fi
}%
\providecommand \natexlab [1]{#1}%
\providecommand \enquote  [1]{``#1''}%
\providecommand \bibnamefont  [1]{#1}%
\providecommand \bibfnamefont [1]{#1}%
\providecommand \citenamefont [1]{#1}%
\providecommand \href@noop [0]{\@secondoftwo}%
\providecommand \href [0]{\begingroup \@sanitize@url \@href}%
\providecommand \@href[1]{\@@startlink{#1}\@@href}%
\providecommand \@@href[1]{\endgroup#1\@@endlink}%
\providecommand \@sanitize@url [0]{\catcode `\\12\catcode `\$12\catcode
  `\&12\catcode `\#12\catcode `\^12\catcode `\_12\catcode `\%12\relax}%
\providecommand \@@startlink[1]{}%
\providecommand \@@endlink[0]{}%
\providecommand \url  [0]{\begingroup\@sanitize@url \@url }%
\providecommand \@url [1]{\endgroup\@href {#1}{\urlprefix }}%
\providecommand \urlprefix  [0]{URL }%
\providecommand \Eprint [0]{\href }%
\providecommand \doibase [0]{http://dx.doi.org/}%
\providecommand \selectlanguage [0]{\@gobble}%
\providecommand \bibinfo  [0]{\@secondoftwo}%
\providecommand \bibfield  [0]{\@secondoftwo}%
\providecommand \translation [1]{[#1]}%
\providecommand \BibitemOpen [0]{}%
\providecommand \bibitemStop [0]{}%
\providecommand \bibitemNoStop [0]{.\EOS\space}%
\providecommand \EOS [0]{\spacefactor3000\relax}%
\providecommand \BibitemShut  [1]{\csname bibitem#1\endcsname}%
\let\auto@bib@innerbib\@empty
%</preamble>
\bibitem [{\citenamefont {Richardson}(1948)}]{Richardson194}%
  \BibitemOpen
  \bibfield  {author} {\bibinfo {author} {\bibfnamefont {L.~F.}\ \bibnamefont
  {Richardson}},\ }\href@noop {} {\bibfield  {journal} {\bibinfo  {journal}
  {Journal of the American Statistical Association}\ }\textbf {\bibinfo
  {volume} {43}},\ \bibinfo {pages} {523} (\bibinfo {year} {1948})}\BibitemShut
  {NoStop}%
\bibitem [{\citenamefont {Morgenstern}\ \emph {et~al.}(2013)\citenamefont
  {Morgenstern}, \citenamefont {Velasquez}, \citenamefont {Manrique},
  \citenamefont {Hong}, \citenamefont {Johnson},\ and\ \citenamefont
  {Johnson}}]{Morgenstern2013}%
  \BibitemOpen
  \bibfield  {author} {\bibinfo {author} {\bibfnamefont {A.~P.}\ \bibnamefont
  {Morgenstern}}, \bibinfo {author} {\bibfnamefont {N.}~\bibnamefont
  {Velasquez}}, \bibinfo {author} {\bibfnamefont {P.}~\bibnamefont {Manrique}},
  \bibinfo {author} {\bibfnamefont {Q.}~\bibnamefont {Hong}}, \bibinfo {author}
  {\bibfnamefont {N.}~\bibnamefont {Johnson}}, \ and\ \bibinfo {author}
  {\bibfnamefont {N.}~\bibnamefont {Johnson}},\ }\href@noop {} {\bibfield
  {journal} {\bibinfo  {journal} {American Journal of Physics}\ }\textbf
  {\bibinfo {volume} {81}},\ \bibinfo {pages} {805} (\bibinfo {year}
  {2013})}\BibitemShut {NoStop}%
\bibitem [{\citenamefont {Cederman}(2003)}]{Cederman2003}%
  \BibitemOpen
  \bibfield  {author} {\bibinfo {author} {\bibfnamefont {L.-E.}\ \bibnamefont
  {Cederman}},\ }\href@noop {} {\bibfield  {journal} {\bibinfo  {journal}
  {American Political Science Review}\ }\textbf {\bibinfo {volume} {97}},\
  \bibinfo {pages} {135} (\bibinfo {year} {2003})}\BibitemShut {NoStop}%
\bibitem [{\citenamefont {Clauset}\ \emph {et~al.}(2009)\citenamefont
  {Clauset}, \citenamefont {Shalizi},\ and\ \citenamefont
  {Newman}}]{Clauset2009}%
  \BibitemOpen
  \bibfield  {author} {\bibinfo {author} {\bibfnamefont {A.}~\bibnamefont
  {Clauset}}, \bibinfo {author} {\bibfnamefont {C.~R.}\ \bibnamefont
  {Shalizi}}, \ and\ \bibinfo {author} {\bibfnamefont {M.~E.~J.}\ \bibnamefont
  {Newman}},\ }\href@noop {} {\bibfield  {journal} {\bibinfo  {journal} {SIAM
  Review}\ }\textbf {\bibinfo {volume} {51}},\ \bibinfo {pages} {661} (\bibinfo
  {year} {2009})}\BibitemShut {NoStop}%
\bibitem [{\citenamefont {Clauset}\ \emph {et~al.}(2007)\citenamefont
  {Clauset}, \citenamefont {Young},\ and\ \citenamefont {Gleditsch}}]{Clauset}%
  \BibitemOpen
  \bibfield  {author} {\bibinfo {author} {\bibfnamefont {A.}~\bibnamefont
  {Clauset}}, \bibinfo {author} {\bibfnamefont {M.}~\bibnamefont {Young}}, \
  and\ \bibinfo {author} {\bibfnamefont {K.}~\bibnamefont {Gleditsch}},\
  }\href@noop {} {\bibfield  {journal} {\bibinfo  {journal} {Journal of
  Conflict Resolution}\ }\textbf {\bibinfo {volume} {51}},\ \bibinfo {pages}
  {58} (\bibinfo {year} {2007})}\BibitemShut {NoStop}%
\bibitem [{\citenamefont {Bohorquez}\ \emph {et~al.}(2009)\citenamefont
  {Bohorquez}, \citenamefont {Gourley}, \citenamefont {Dixon}, \citenamefont
  {Spagat},\ and\ \citenamefont {Johnson}}]{Bohorquez2009a}%
  \BibitemOpen
  \bibfield  {author} {\bibinfo {author} {\bibfnamefont {J.~C.}\ \bibnamefont
  {Bohorquez}}, \bibinfo {author} {\bibfnamefont {S.}~\bibnamefont {Gourley}},
  \bibinfo {author} {\bibfnamefont {A.~R.}\ \bibnamefont {Dixon}}, \bibinfo
  {author} {\bibfnamefont {M.}~\bibnamefont {Spagat}}, \ and\ \bibinfo {author}
  {\bibfnamefont {N.~F.}\ \bibnamefont {Johnson}},\ }\href@noop {} {\bibfield
  {journal} {\bibinfo  {journal} {Nature}\ }\textbf {\bibinfo {volume} {462}},\
  \bibinfo {pages} {911} (\bibinfo {year} {2009})}\BibitemShut {NoStop}%
\bibitem [{\citenamefont {Dedeo}\ \emph {et~al.}(2011)\citenamefont {Dedeo},
  \citenamefont {Krakauer},\ and\ \citenamefont {Flack}}]{Dedeo2011}%
  \BibitemOpen
  \bibfield  {author} {\bibinfo {author} {\bibfnamefont {S.}~\bibnamefont
  {Dedeo}}, \bibinfo {author} {\bibfnamefont {D.}~\bibnamefont {Krakauer}}, \
  and\ \bibinfo {author} {\bibfnamefont {J.}~\bibnamefont {Flack}},\
  }\href@noop {} {\bibfield  {journal} {\bibinfo  {journal} {Journal of the
  Royal Society Interface}\ }\textbf {\bibinfo {volume} {8}},\ \bibinfo {pages}
  {1260} (\bibinfo {year} {2011})}\BibitemShut {NoStop}%
\bibitem [{\citenamefont {Barab\'{a}si}(2005)}]{Barabasi2005}%
  \BibitemOpen
  \bibfield  {author} {\bibinfo {author} {\bibfnamefont {A.-L.}\ \bibnamefont
  {Barab\'{a}si}},\ }\href@noop {} {\bibfield  {journal} {\bibinfo  {journal}
  {Nature}\ }\textbf {\bibinfo {volume} {435}},\ \bibinfo {pages} {207}
  (\bibinfo {year} {2005})}\BibitemShut {NoStop}%
\bibitem [{\citenamefont {Gabaix}\ \emph {et~al.}(2003)\citenamefont {Gabaix},
  \citenamefont {Gopikrishnan}, \citenamefont {Plerou},\ and\ \citenamefont
  {Stanley}}]{Gabaix2003}%
  \BibitemOpen
  \bibfield  {author} {\bibinfo {author} {\bibfnamefont {X.}~\bibnamefont
  {Gabaix}}, \bibinfo {author} {\bibfnamefont {P.}~\bibnamefont
  {Gopikrishnan}}, \bibinfo {author} {\bibfnamefont {V.}~\bibnamefont
  {Plerou}}, \ and\ \bibinfo {author} {\bibfnamefont {H.~E.}\ \bibnamefont
  {Stanley}},\ }\href@noop {} {\bibfield  {journal} {\bibinfo  {journal}
  {Nature}\ }\textbf {\bibinfo {volume} {423}},\ \bibinfo {pages} {267}
  (\bibinfo {year} {2003})}\BibitemShut {NoStop}%
\bibitem [{\citenamefont {Johnson}\ \emph {et~al.}(2013)\citenamefont
  {Johnson}, \citenamefont {Medina}, \citenamefont {Zhao}, \citenamefont
  {Messinger}, \citenamefont {Horgan}, \citenamefont {Gill}, \citenamefont
  {Bohorquez}, \citenamefont {Mattson}, \citenamefont {Gangi}, \citenamefont
  {Qi}, \citenamefont {Manrique}, \citenamefont {Velasquez}, \citenamefont
  {Morgenstern}, \citenamefont {Restrepo}, \citenamefont {Johnson},
  \citenamefont {Spagat},\ and\ \citenamefont {Zarama}}]{Johnson2013b}%
  \BibitemOpen
  \bibfield  {author} {\bibinfo {author} {\bibfnamefont {N.~F.}\ \bibnamefont
  {Johnson}}, \bibinfo {author} {\bibfnamefont {P.}~\bibnamefont {Medina}},
  \bibinfo {author} {\bibfnamefont {G.}~\bibnamefont {Zhao}}, \bibinfo {author}
  {\bibfnamefont {D.~S.}\ \bibnamefont {Messinger}}, \bibinfo {author}
  {\bibfnamefont {J.}~\bibnamefont {Horgan}}, \bibinfo {author} {\bibfnamefont
  {P.}~\bibnamefont {Gill}}, \bibinfo {author} {\bibfnamefont {J.~C.}\
  \bibnamefont {Bohorquez}}, \bibinfo {author} {\bibfnamefont {W.}~\bibnamefont
  {Mattson}}, \bibinfo {author} {\bibfnamefont {D.}~\bibnamefont {Gangi}},
  \bibinfo {author} {\bibfnamefont {H.}~\bibnamefont {Qi}}, \bibinfo {author}
  {\bibfnamefont {P.}~\bibnamefont {Manrique}}, \bibinfo {author}
  {\bibfnamefont {N.}~\bibnamefont {Velasquez}}, \bibinfo {author}
  {\bibfnamefont {A.}~\bibnamefont {Morgenstern}}, \bibinfo {author}
  {\bibfnamefont {E.}~\bibnamefont {Restrepo}}, \bibinfo {author}
  {\bibfnamefont {N.}~\bibnamefont {Johnson}}, \bibinfo {author} {\bibfnamefont
  {M.}~\bibnamefont {Spagat}}, \ and\ \bibinfo {author} {\bibfnamefont
  {R.}~\bibnamefont {Zarama}},\ }\href@noop {} {\bibfield  {journal} {\bibinfo
  {journal} {Scientific Reports}\ }\textbf {\bibinfo {volume} {3}},\ \bibinfo
  {pages} {3463} (\bibinfo {year} {2013})}\BibitemShut {NoStop}%
\bibitem [{\citenamefont {Johnson}\ \emph {et~al.}(2011)\citenamefont
  {Johnson}, \citenamefont {Carran}, \citenamefont {Botner}, \citenamefont
  {Fontaine}, \citenamefont {Laxague}, \citenamefont {Nuetzel}, \citenamefont
  {Turnley},\ and\ \citenamefont {Tivnan}}]{Johnson2011a}%
  \BibitemOpen
  \bibfield  {author} {\bibinfo {author} {\bibfnamefont {N.}~\bibnamefont
  {Johnson}}, \bibinfo {author} {\bibfnamefont {S.}~\bibnamefont {Carran}},
  \bibinfo {author} {\bibfnamefont {J.}~\bibnamefont {Botner}}, \bibinfo
  {author} {\bibfnamefont {K.}~\bibnamefont {Fontaine}}, \bibinfo {author}
  {\bibfnamefont {N.}~\bibnamefont {Laxague}}, \bibinfo {author} {\bibfnamefont
  {P.}~\bibnamefont {Nuetzel}}, \bibinfo {author} {\bibfnamefont
  {J.}~\bibnamefont {Turnley}}, \ and\ \bibinfo {author} {\bibfnamefont
  {B.}~\bibnamefont {Tivnan}},\ }\href@noop {} {\bibfield  {journal} {\bibinfo
  {journal} {Science Magazine}\ }\textbf {\bibinfo {volume} {333}},\ \bibinfo
  {pages} {81} (\bibinfo {year} {2011})}\BibitemShut {NoStop}%
\bibitem [{\citenamefont {Hall}\ and\ \citenamefont {Coyne}(2013)}]{wars}%
  \BibitemOpen
  \bibfield  {author} {\bibinfo {author} {\bibfnamefont {A.~R.}\ \bibnamefont
  {Hall}}\ and\ \bibinfo {author} {\bibfnamefont {C.~J.}\ \bibnamefont
  {Coyne}},\ }\href@noop {} {\bibfield  {journal} {\bibinfo  {journal} {Defence
  and Peace Economics}\ }\textbf {\bibinfo {volume} {25}},\ \bibinfo {pages}
  {445} (\bibinfo {year} {2013})}\BibitemShut {NoStop}%
\bibitem [{\citenamefont {Johnson}\ \emph {et~al.}(2006)\citenamefont
  {Johnson}, \citenamefont {Spagat}, \citenamefont {Restrepo}, \citenamefont
  {Becerra}, \citenamefont {Camilo}, \citenamefont {Su\'{a}rez}, \citenamefont
  {Restrepo},\ and\ \citenamefont {Zarama}}]{Johnson}%
  \BibitemOpen
  \bibfield  {author} {\bibinfo {author} {\bibfnamefont {N.~F.}\ \bibnamefont
  {Johnson}}, \bibinfo {author} {\bibfnamefont {M.}~\bibnamefont {Spagat}},
  \bibinfo {author} {\bibfnamefont {J.~A.}\ \bibnamefont {Restrepo}}, \bibinfo
  {author} {\bibfnamefont {O.}~\bibnamefont {Becerra}}, \bibinfo {author}
  {\bibfnamefont {J.}~\bibnamefont {Camilo}}, \bibinfo {author} {\bibfnamefont
  {N.}~\bibnamefont {Su\'{a}rez}}, \bibinfo {author} {\bibfnamefont {E.~M.}\
  \bibnamefont {Restrepo}}, \ and\ \bibinfo {author} {\bibfnamefont
  {R.}~\bibnamefont {Zarama}},\ }\href@noop {} {\bibfield  {journal} {\bibinfo
  {journal} {arXiv:physics/0605035}\ } (\bibinfo {year} {2006})}\BibitemShut
  {NoStop}%
\bibitem [{\citenamefont {Clauset}\ and\ \citenamefont
  {Gleditsch}(2012)}]{Clauset2012}%
  \BibitemOpen
  \bibfield  {author} {\bibinfo {author} {\bibfnamefont {A.}~\bibnamefont
  {Clauset}}\ and\ \bibinfo {author} {\bibfnamefont {K.~S.}\ \bibnamefont
  {Gleditsch}},\ }\href@noop {} {\bibfield  {journal} {\bibinfo  {journal}
  {PloS ONE}\ }\textbf {\bibinfo {volume} {7}},\ \bibinfo {pages} {e48633}
  (\bibinfo {year} {2012})}\BibitemShut {NoStop}%
\bibitem [{\citenamefont {Havlin}\ and\ \citenamefont
  {Ben-Avraham}(1982)}]{Havlin1982}%
  \BibitemOpen
  \bibfield  {author} {\bibinfo {author} {\bibfnamefont {S.}~\bibnamefont
  {Havlin}}\ and\ \bibinfo {author} {\bibfnamefont {D.}~\bibnamefont
  {Ben-Avraham}},\ }\href@noop {} {\bibfield  {journal} {\bibinfo  {journal}
  {Physical Review A}\ }\textbf {\bibinfo {volume} {26}},\ \bibinfo {pages}
  {1728} (\bibinfo {year} {1982})}\BibitemShut {NoStop}%
\end{thebibliography}%

\clearpage

\end{document}
